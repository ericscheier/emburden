\documentclass[
]{article}

%% recommended packages
\usepackage{orcidlink,thumbpdf,lmodern}

\usepackage[utf8]{inputenc}

\author{
Eric Scheier\\Independent Researcher
}
\title{}

\Plainauthor{Eric Scheier}


\Abstract{
Energy burden---the proportion of household income spent on energy---is
a critical metric for understanding energy poverty and inequity.
However, traditional energy burden ratios present analytical challenges
including difficulties with aggregation and visualization of extreme
values. The \pkg{emburden} package for \proglang{R} implements Net
Energy Return (Nh) methodology to address these limitations while
enabling temporal analysis of household energy characteristics. This
paper introduces the package's design and demonstrates its application
to comparing Low-Income Energy Affordability Data (LEAD) Tool vintages
from 2018 and 2022 across geographic and demographic dimensions. The
package provides functions for downloading, processing, and analyzing
census tract-level energy burden data for all U.S. states, with
particular attention to proper weighted aggregation and schema
normalization across data vintages. We demonstrate the package's
capabilities through examples ranging from state-level summaries to
fine-grained census tract comparisons, illustrating how policy-relevant
insights can be extracted at multiple scales.
}


%% publication information
%% \Volume{50}
%% \Issue{9}
%% \Month{June}
%% \Year{2012}
%% \Submitdate{}
%% \Acceptdate{2012-06-04}

\Address{
    Eric Scheier\\
    Independent Researcher\\
    Durham, North Carolina\\
  E-mail: \href{mailto:eric@scheier.org}{\nolinkurl{eric@scheier.org}}\\
  URL: \url{https://github.com/ericscheier}\\~\\
  }


% tightlist command for lists without linebreak
\providecommand{\tightlist}{%
  \setlength{\itemsep}{0pt}\setlength{\parskip}{0pt}}




\usepackage{amsmath}

\begin{document}



\section{Introduction}\label{introduction}

Household energy affordability is a persistent challenge affecting
millions of households in the United States. Low-income households face
disproportionate energy burdens, often spending more than 6\% of their
income on energy costs compared to 2-3\% for higher-income households
\citep{ross2018high, drehobl2016lifting}. Understanding these
disparities and tracking changes over time is essential for designing
effective energy assistance programs and policies.

The traditional energy burden metric---the ratio of energy expenditures
(\(S\)) to gross income (\(G\))---has several analytical limitations. As
a ratio with income in the denominator, energy burden (\(E_b = S/G\))
approaches infinity for households with very low incomes, creating
challenges for aggregation and visualization. Additionally, the metric
requires harmonic mean aggregation rather than arithmetic means, which
is not widely understood or consistently applied
\citep{scheier2022measurement}.

\subsection{Mathematical foundations}\label{mathematical-foundations}

The \pkg{emburden} package for \proglang{R} addresses these challenges
by implementing Net Energy Return (NER) methodology, adapted from
macro-energy systems analysis
\citep{hall2011eroi, brandtcalculating2013, carbajalesdale2014better}.
Net energy analysis estimates the net energy return of a process as a
relationship between gross resources extracted and embodied energy
directed toward extraction:

\[G = Gross\ Resource\ Extracted\]

\[S = Spending\ on\ Extraction\ Process\]

\[Net\ Energy\ Return\ (NER) = \frac{G - S}{S}\]

For households extracting income from the economy, these ratios become:

\[G_{income} = Gross\ Income\]

\[S_{energy} = Spending\ on\ Energy\]

\[NER_{household} = \frac{G_{income} - S_{energy}}{S_{energy}}\]

This metric represents the net earnings a household receives for every
dollar of expenditure on secondary energy. For notational simplicity, we
use \(N_h\) to denote household Net Energy Return throughout this paper,
where \(N_h = NER_{household}\).

\subsubsection{Comparison with energy
burden}\label{comparison-with-energy-burden}

Energy burden, the traditional metric in energy poverty analysis, is
defined as:

\[Energy\ Burden = E_b = \frac{S_{energy}}{G_{income}}\]

While energy burden is intuitive as a percentage, it has several
mathematical limitations. The Net Energy Return transformation addresses
these by preventing double-counting of energy expenditures (income in
the numerator already includes the portion spent on energy) and enabling
proper weighted mean aggregation:

\[\overline{N_h} = \frac{\sum (N_h \times households)}{\sum households}\]

In contrast, energy burden requires harmonic mean aggregation:

\[\overline{E_b} = \frac{1}{\overline{1/E_b}}\]

The two metrics are mathematically related through the transformation
\(E_b = 1/(N_h + 1)\), allowing seamless conversion between
representations.

\subsubsection{Energy poverty threshold}\label{energy-poverty-threshold}

Energy poverty is commonly defined as spending greater than 10\% of
household income on energy \citep{bednarrecognition2020}:

\[E_b^{*} = \frac{S_{energy}}{G_{income}} > 10\%\]

Translated to Net Energy Return, the energy poverty threshold becomes:

\[N_h^{*} < 9: Household\ at\ Energy\ Poverty\ Line\]

This means a household earning less than \$9 of income for every dollar
spent on secondary energy is considered to be in energy poverty by the
traditional energy burden accounting method. A Net Energy Return of 9 or
lower is equivalent to an energy burden of 10\% or higher. While this
threshold is somewhat arbitrary and may not be suitable in all
situations, it provides a useful benchmark for comparing results to the
energy poverty literature.

\subsection{The LEAD Tool and temporal
analysis}\label{the-lead-tool-and-temporal-analysis}

The U.S. Department of Energy's Low-Income Energy Affordability Data
(LEAD) Tool \citep{ma2019lowincome} provides census tract-level
estimates of household energy characteristics based on American
Community Survey microdata. The tool uses iterative proportional fitting
to allocate households to census tracts while calibrating to
utility-reported sales and revenues.

Multiple vintages of LEAD Tool data have been released:

\begin{itemize}
\tightlist
\item
  \textbf{2018 Update}: Based on 2018 5-year ACS data, released July
  2020
\item
  \textbf{2022 Update}: Based on 2022 5-year ACS data, released August
  2024
\end{itemize}

These vintages enable temporal analysis of energy burden trends, but
require careful handling of schema differences and income bracket
definitions.

\subsection{Package design philosophy}\label{package-design-philosophy}

The \pkg{emburden} package is designed around several key principles:

\begin{enumerate}
\def\labelenumi{\arabic{enumi}.}
\tightlist
\item
  \textbf{Proper aggregation}: Implements weighted mean aggregation
  using Net Energy Return, with household counts as weights
\item
  \textbf{Temporal consistency}: Normalizes schema differences between
  LEAD Tool vintages to enable valid comparisons
\item
  \textbf{Flexible workflows}: Supports both database and CSV-based data
  access with automatic fallback
\item
  \textbf{Geographic flexibility}: Enables analysis from national level
  down to individual census tracts
\end{enumerate}

\section{Methodology}\label{methodology}

\subsection{Data sources}\label{data-sources}

The \pkg{emburden} package provides access to three primary datasets for
household energy burden analysis:

\subsubsection{LEAD Tool}\label{lead-tool}

The Low-Income Energy Affordability Data (LEAD) Tool
\citep{ma2019lowincome} portrays average income, electricity
expenditures, gas expenditures, and other fuel expenditures for cohorts
of households segmented by location (census tract, county, state) and
household characteristics (ownership status, building age, number of
units, attachment status, primary heating fuel).

The dataset is assembled using iterative proportional fitting (IPF), a
widely used spatial microsimulation method to allocate households to
census tracts while calibrating characteristics to known quantities. The
IPF algorithm processes cross-tabulations of household responses from
the American Community Survey (ACS) Public Use Microdata Samples,
scaling them to match aggregate annual values from utility sales and
revenues reported in Energy Information Administration forms 861
(electricity) and 176 (natural gas).

Multiple vintages are available:

\begin{itemize}
\tightlist
\item
  \textbf{2018 Update}: Based on 2016 5-year ACS data (2012-2016),
  released July 2020
\item
  \textbf{2022 Update}: Based on 2018 5-year ACS data (2014-2018),
  released August 2024
\end{itemize}

\subsubsection{REPLICA dataset}\label{replica-dataset}

The Renewable Energy Potential of Low-Income Communities in America
(REPLICA) dataset \citep{sigrinRooftopSolarTechnical2018} adds technical
rooftop solar potential and additional techno-economic variables
including demographics and electricity rates. The package can merge
REPLICA data with LEAD data to enrich analyses with utility type, locale
classification, and solar generation potential.

\subsubsection{Schema normalization across
vintages}\label{schema-normalization-across-vintages}

A critical challenge in temporal analysis is handling schema differences
between LEAD Tool vintages. The package implements automatic
normalization through the following transformations:

\textbf{Income bracket aggregation}: The LEAD Tool provides income as a
fraction of Area Median Income (AMI) or Federal Poverty Level (FPL). For
AMI data, the package can aggregate detailed brackets into simplified
categories matching the REPLICA schema:

\begin{itemize}
\tightlist
\item
  0-30\% AMI: Very Low Income
\item
  30-80\% AMI: Low-to-Moderate Income
\item
  ≥80\% AMI: Middle-to-High Income
\end{itemize}

For FPL data, the aggregation follows poverty line definitions:

\begin{itemize}
\tightlist
\item
  0-100\% FPL: In Poverty
\item
  ≥100\% FPL: Not In Poverty
\end{itemize}

\textbf{Building type simplification}: Housing units are classified as:

\begin{itemize}
\item
  1 Unit: Single-Family
\item
  \begin{quote}
  1 Unit: Multi-Family
  \end{quote}
\item
  Other Unit: Excluded from analysis
\end{itemize}

These normalizations enable valid temporal comparisons despite
underlying schema evolution between vintages.

\subsection{Data processing}\label{data-processing}

The package processes raw LEAD Tool data through several stages:

\subsubsection{Energy burden indicator
calculation}\label{energy-burden-indicator-calculation}

For each household cohort, the package calculates:

\[s = electricity + natural\ gas + other\ fuels\]

\[g = annual\ household\ income\]

From these base metrics, all energy burden indicators are derived using
the formulas presented in Section 1.1.

\subsubsection{Weighted aggregation}\label{weighted-aggregation}

The package implements proper weighted aggregation using household
counts as weights. For Net Energy Return:

\begin{CodeChunk}
\begin{CodeInput}
R> calculate_weighted_metrics(
+   data,
+   group_columns = c("state", "income_bracket"),
+   metric_name = "ner"
+ )
\end{CodeInput}
\end{CodeChunk}

This function:

\begin{enumerate}
\def\labelenumi{\arabic{enumi}.}
\tightlist
\item
  Filters data to specified groups
\item
  Calculates weighted means using household counts
\item
  Computes poverty rates below specified thresholds
\item
  Returns summary statistics including quantiles and standard deviations
\end{enumerate}

The key insight is that Net Energy Return allows arithmetic weighted
means, while energy burden would require harmonic mean aggregation---a
distinction that significantly impacts the validity and interpretability
of aggregate statistics.

\subsubsection{Data quality
considerations}\label{data-quality-considerations}

Iterative proportional fitting has limitations as an estimation
procedure. The relationship between constraint variables tends toward
the average of the initializing dataset, potentially depressing
variations among otherwise similar regions. This may explain the large
quantities of households estimated to have very low incomes. Validating
these estimated data would require randomized surveys along the
dimensions of interest.

Additionally, the ``primary heating fuel'' category derives from the ACS
question ``Which fuel is used most for heating this house, apartment, or
mobile home?'' The predictive power of this question for energy
expenditures is not fully understood and warrants caution in
interpretation.

Though REPLICA relies on a different LEAD vintage (2017) than recent
analyses (2019, 2022), the package still enables useful cross-dataset
analysis. However, inferring differences among annual estimates should
account for the standard error of the data \citep{ma2019lowincome}.
Rigorous temporal analysis benefits from comparing identically-processed
vintages.

\section{Package architecture}\label{package-architecture}

The \pkg{emburden} package is organized into several functional modules:

\subsection{Core functions}\label{core-functions}

\begin{CodeChunk}
\begin{CodeInput}
R> library(emburden)
R> 
R> # Energy metric calculations
R> energy_burden_func(gross_income, energy_spending)
R> ner_func(gross_income, energy_spending)  # Net Energy Return
R> eroi_func(gross_income, energy_spending)  # EROI
R> dear_func(gross_income, energy_spending)  # DEAR
R> 
R> # Statistical aggregation
R> calculate_weighted_metrics(
+   graph_data,
+   group_columns = "state",
+   metric_name = "ner"
+ )
\end{CodeInput}
\end{CodeChunk}

\subsection{Data loading functions}\label{data-loading-functions}

The package provides automatic data downloading and caching:

\begin{CodeChunk}
\begin{CodeInput}
R> # Load census tract data (auto-downloads if not available)
R> nc_tracts <- load_census_tract_data(states = "NC")
R> 
R> # Load cohort data by income bracket
R> nc_ami <- load_cohort_data(
+   dataset = "ami",
+   states = "NC",
+   vintage = "2022"
+ )
R> 
R> # Compare vintages
R> comparison <- compare_energy_burden(
+   dataset = "ami",
+   states = "NC",
+   group_by = "state"
+ )
\end{CodeInput}
\end{CodeChunk}

\section{Analysis examples}\label{analysis-examples}

The \pkg{emburden} package's primary contribution is enabling temporal
analysis of energy burden through proper schema normalization and
aggregation. This section demonstrates the package's capabilities
through progressively detailed examples.

\subsection{Temporal comparison
workflow}\label{temporal-comparison-workflow}

The \texttt{compare\_energy\_burden()} function provides the core
temporal analysis functionality:

\begin{CodeChunk}
\begin{CodeInput}
R> library(emburden)
R> 
R> # Compare North Carolina energy burden: 2018 vs 2022
R> nc_comparison <- compare_energy_burden(
+   dataset = "ami",
+   states = "NC",
+   group_by = "income_bracket"
+ )
R> 
R> # View formatted comparison table
R> print(nc_comparison)
\end{CodeInput}
\end{CodeChunk}

The function automatically:

\begin{enumerate}
\def\labelenumi{\arabic{enumi}.}
\tightlist
\item
  Downloads both vintages if not cached locally
\item
  Normalizes schema differences between vintages
\item
  Performs proper \(N_h\)-based weighted aggregation
\item
  Calculates energy burden for both periods
\item
  Computes changes in percentage points
\end{enumerate}

\subsubsection{Understanding the output}\label{understanding-the-output}

The comparison object contains multiple metrics:

\begin{CodeChunk}
\begin{CodeInput}
R> # Energy burden in 2018 and 2022
R> nc_comparison$neb_2018
R> nc_comparison$neb_2022
R> 
R> # Change in energy burden (percentage points)
R> nc_comparison$neb_change_pp
R> 
R> # Net Energy Return values
R> nc_comparison$ner_2018
R> nc_comparison$ner_2022
R> 
R> # Household counts
R> nc_comparison$households_2018
R> nc_comparison$households_2022
\end{CodeInput}
\end{CodeChunk}

\subsection{Example 1: State-level temporal
analysis}\label{example-1-state-level-temporal-analysis}

To examine overall state changes without grouping by demographic
characteristics:

\begin{CodeChunk}
\begin{CodeInput}
R> # Overall state comparison
R> nc_state <- compare_energy_burden(
+   dataset = "ami",
+   states = "NC",
+   group_by = "none"
+ )
R> 
R> # Extract key findings
R> cat(sprintf(
+   "North Carolina energy burden changed from %.1f%% (2018) to %.1f%% (2022)\n",
+   nc_state$neb_2018 * 100,
+   nc_state$neb_2022 * 100
+ ))
R> 
R> cat(sprintf(
+   "Change: %+.2f percentage points\n",
+   nc_state$neb_change_pp * 100
+ ))
\end{CodeInput}
\end{CodeChunk}

\subsection{Example 2: Income bracket
analysis}\label{example-2-income-bracket-analysis}

Disaggregating by income bracket reveals which populations experienced
the largest changes:

\begin{CodeChunk}
\begin{CodeInput}
R> # Compare by income bracket
R> nc_income <- compare_energy_burden(
+   dataset = "ami",
+   states = "NC",
+   group_by = "income_bracket"
+ )
R> 
R> # Visualize changes
R> library(ggplot2)
R> 
R> ggplot(nc_income, aes(x = income_bracket, y = neb_change_pp * 100)) +
+   geom_col(fill = "steelblue") +
+   geom_hline(yintercept = 0, linetype = "dashed") +
+   labs(
+     title = "Change in Energy Burden by Income Bracket",
+     subtitle = "North Carolina, 2018 to 2022",
+     x = "Income Bracket (% of Area Median Income)",
+     y = "Change in Energy Burden (percentage points)"
+   ) +
+   theme_minimal()
\end{CodeInput}
\end{CodeChunk}

Typical findings show that very low-income households (0-30\% AMI)
experience the highest energy burdens and are most vulnerable to changes
in energy costs or income levels.

\subsection{Example 3: Multi-state
comparison}\label{example-3-multi-state-comparison}

Comparing multiple states reveals regional patterns and policy impacts:

\begin{CodeChunk}
\begin{CodeInput}
R> # Compare Southern states
R> southern_states <- compare_energy_burden(
+   dataset = "ami",
+   states = c("NC", "SC", "GA", "FL"),
+   group_by = "state"
+ )
R> 
R> # Which states improved most?
R> southern_states %>%
+   arrange(neb_change_pp) %>%
+   select(state_abbr, neb_2018, neb_2022, neb_change_pp)
R> 
R> # Visualize state comparison
R> ggplot(southern_states, aes(x = reorder(state_abbr, neb_2022),
+                              y = neb_2022 * 100)) +
+   geom_col(fill = "darkgreen") +
+   geom_point(aes(y = neb_2018 * 100), color = "red", size = 3) +
+   labs(
+     title = "Energy Burden by State: 2022 (bars) vs 2018 (points)",
+     x = "State",
+     y = "Energy Burden (%)"
+   ) +
+   theme_minimal()
\end{CodeInput}
\end{CodeChunk}

\subsection{Example 4: Housing tenure
analysis}\label{example-4-housing-tenure-analysis}

Energy burden often varies significantly between renters and homeowners:

\begin{CodeChunk}
\begin{CodeInput}
R> # Compare by housing tenure
R> nc_tenure <- compare_energy_burden(
+   dataset = "ami",
+   states = "NC",
+   group_by = "housing_tenure"
+ )
R> 
R> # Calculate the renter-owner gap
R> gap_2018 <- nc_tenure$neb_2018[nc_tenure$housing_tenure == "RENTER"] -
+             nc_tenure$neb_2018[nc_tenure$housing_tenure == "OWNER"]
R> 
R> gap_2022 <- nc_tenure$neb_2022[nc_tenure$housing_tenure == "RENTER"] -
+             nc_tenure$neb_2022[nc_tenure$housing_tenure == "OWNER"]
R> 
R> cat(sprintf(
+   "Renter-Owner energy burden gap: %.2f pp (2018) → %.2f pp (2022)\n",
+   gap_2018 * 100,
+   gap_2022 * 100
+ ))
\end{CodeInput}
\end{CodeChunk}

Renters typically face higher energy burdens due to split-incentive
problems where landlords make efficiency investment decisions but
tenants pay energy bills.

\subsection{Example 5: Federal Poverty Line
analysis}\label{example-5-federal-poverty-line-analysis}

For policy applications targeting households below the federal poverty
line:

\begin{CodeChunk}
\begin{CodeInput}
R> # Use FPL dataset instead of AMI
R> nc_fpl <- compare_energy_burden(
+   dataset = "fpl",
+   states = "NC",
+   group_by = "income_bracket"
+ )
R> 
R> # Compare poverty vs non-poverty households
R> nc_fpl %>%
+   filter(income_bracket %in% c("Below Federal Poverty Line",
+                                 "Above Federal Poverty Line")) %>%
+   select(income_bracket, neb_2018, neb_2022, neb_change_pp)
\end{CodeInput}
\end{CodeChunk}

This analysis is particularly relevant for programs like the Low-Income
Home Energy Assistance Program (LIHEAP) which target households below
specific poverty thresholds.

\subsection{Example 6: Census tract-level
analysis}\label{example-6-census-tract-level-analysis}

For fine-grained spatial analysis, load tract-level data directly:

\begin{CodeChunk}
\begin{CodeInput}
R> # Load 2022 census tract data
R> nc_tracts_2022 <- load_census_tract_data(
+   states = "NC",
+   vintage = "2022"
+ )
R> 
R> # Calculate county-level statistics
R> nc_counties <- calculate_weighted_metrics(
+   nc_tracts_2022,
+   group_columns = "county_name",
+   metric_name = "ner"
+ )
R> 
R> # Identify counties with highest energy burden
R> nc_counties %>%
+   mutate(energy_burden = 1 / (ner + 1)) %>%
+   arrange(desc(energy_burden)) %>%
+   head(10) %>%
+   select(county_name, energy_burden, household_count)
\end{CodeInput}
\end{CodeChunk}

Census tract data enables spatial analysis and mapping applications,
revealing urban-rural disparities and identifying communities in need of
targeted assistance.

\section{Discussion}\label{discussion}

\subsection{Policy implications}\label{policy-implications}

The ability to track energy burden changes over time has important
policy implications. Programs like LIHEAP (Low-Income Home Energy
Assistance Program) and WAP (Weatherization Assistance Program) target
households experiencing energy insecurity, but evaluating their
effectiveness requires robust temporal analysis.

The \pkg{emburden} package enables researchers and policymakers to:

\begin{enumerate}
\def\labelenumi{\arabic{enumi}.}
\tightlist
\item
  \textbf{Track program impacts}: Compare energy burden before and after
  policy interventions
\item
  \textbf{Identify vulnerable populations}: Disaggregate trends by
  income, tenure, and geography
\item
  \textbf{Allocate resources effectively}: Target communities with
  worsening energy affordability
\item
  \textbf{Benchmark across jurisdictions}: Compare state and local
  policy outcomes
\end{enumerate}

\subsubsection{Split-incentive and principal-agent
problems}\label{split-incentive-and-principal-agent-problems}

A persistent challenge in energy equity is the split-incentive problem:
landlords make energy efficiency investment decisions, but tenants pay
the energy bills. This misalignment of incentives leads to
underinvestment in efficiency improvements for rental properties.

The package's ability to analyze energy burden by housing tenure reveals
the magnitude of this problem:

\begin{CodeChunk}
\begin{CodeInput}
R> # Quantify the renter-owner gap
R> tenure_comparison <- compare_energy_burden(
+   dataset = "ami",
+   states = "all",  # National analysis
+   group_by = "housing_tenure"
+ )
R> 
R> # Calculate disparity
R> renter_burden <- tenure_comparison$neb_2022[
+   tenure_comparison$housing_tenure == "RENTER"
+ ]
R> owner_burden <- tenure_comparison$neb_2022[
+   tenure_comparison$housing_tenure == "OWNER"
+ ]
R> 
R> disparity_ratio <- renter_burden / owner_burden
\end{CodeInput}
\end{CodeChunk}

Addressing this gap requires policy interventions such as:

\begin{itemize}
\tightlist
\item
  On-bill financing programs
\item
  Landlord incentive programs
\item
  Energy efficiency standards for rental properties
\item
  Community-scale renewable energy projects
\end{itemize}

\subsection{Data limitations and
considerations}\label{data-limitations-and-considerations}

Users should be aware of several data limitations:

\subsubsection{Iterative proportional fitting
constraints}\label{iterative-proportional-fitting-constraints}

The LEAD Tool uses IPF to allocate households to census tracts, which
has important implications:

\begin{enumerate}
\def\labelenumi{\arabic{enumi}.}
\tightlist
\item
  \textbf{Regression toward the mean}: IPF tends to depress variations
  among similar regions
\item
  \textbf{Estimation uncertainty}: Standard errors are substantial,
  especially for small cohorts
\item
  \textbf{Temporal comparability}: Different ACS vintages may have
  methodological differences
\end{enumerate}

\subsubsection{Income measurement
challenges}\label{income-measurement-challenges}

Household income as reported in the ACS has known limitations:

\begin{itemize}
\tightlist
\item
  \textbf{Underreporting}: Particularly for benefits and informal income
\item
  \textbf{Timing}: Income is annual but energy costs vary seasonally
\item
  \textbf{Household composition}: Per-capita income may be more relevant
  for some analyses
\end{itemize}

\subsubsection{Energy expenditure
estimation}\label{energy-expenditure-estimation}

The ``primary heating fuel'' categorization derives from a single ACS
question and may not fully capture:

\begin{itemize}
\tightlist
\item
  Mixed-fuel households
\item
  Behavioral patterns
\item
  Appliance efficiency variations
\item
  Climate variations within states
\end{itemize}

Despite these limitations, the LEAD Tool represents the most
comprehensive spatial dataset available for energy burden analysis in
the United States.

\subsection{Future research
directions}\label{future-research-directions}

Several extensions would enhance the package's capabilities:

\subsubsection{Additional vintages}\label{additional-vintages}

As DOE releases new LEAD Tool vintages (potentially 2024, 2026, etc.),
the package can incorporate them to enable longer-term trend analysis.
This would support:

\begin{itemize}
\tightlist
\item
  Multi-year trend identification
\item
  Correlation with economic cycles
\item
  Climate change impact assessment
\end{itemize}

\subsubsection{Additional metrics}\label{additional-metrics}

The package currently implements Net Energy Return, EROI, and DEAR.
Future versions could add:

\begin{itemize}
\tightlist
\item
  \textbf{Disposable income ratios}: Accounting for essential expenses
  beyond energy
\item
  \textbf{Energy poverty depth}: How far below thresholds households
  fall
\item
  \textbf{Vulnerability indices}: Combining burden with demographic risk
  factors
\end{itemize}

\subsubsection{Spatial analysis
enhancements}\label{spatial-analysis-enhancements}

Geographic extensions could include:

\begin{itemize}
\tightlist
\item
  Integration with climate zone data
\item
  Utility service territory analysis
\item
  Transportation energy burden incorporation
\item
  Built environment characteristics
\end{itemize}

\subsubsection{Causal analysis tools}\label{causal-analysis-tools}

Methodological extensions for policy evaluation:

\begin{itemize}
\tightlist
\item
  Difference-in-differences estimation
\item
  Synthetic control methods
\item
  Regression discontinuity designs
\item
  Propensity score matching
\end{itemize}

\subsection{Comparison with existing
tools}\label{comparison-with-existing-tools}

Several tools exist for energy burden analysis, each with different
strengths:

\begin{itemize}
\tightlist
\item
  \textbf{LEAD Tool web interface}: Interactive but limited temporal
  comparison
\item
  \textbf{State energy office tools}: Customized but not standardized
  across states
\item
  \textbf{Academic datasets}: Rich but often one-time snapshots
\item
  \textbf{\pkg{emburden}}: Focused on temporal analysis with proper
  aggregation methodology
\end{itemize}

The \pkg{emburden} package fills a gap by providing programmatic access
to multiple vintages with automated schema normalization, enabling
reproducible temporal analyses at scale.

\section{Conclusion}\label{conclusion}

The \pkg{emburden} package provides a robust framework for temporal
analysis of household energy burden using proper Net Energy Return
methodology. By automating data access, normalizing schema differences,
and implementing correct aggregation methods, the package enables
researchers and policymakers to track energy affordability trends across
multiple scales.

Key contributions include:

\begin{enumerate}
\def\labelenumi{\arabic{enumi}.}
\tightlist
\item
  \textbf{Mathematical foundations}: Proper Net Energy Return
  aggregation avoiding double-counting
\item
  \textbf{Temporal consistency}: Automated schema normalization across
  LEAD Tool vintages
\item
  \textbf{Flexible analysis}: Functions supporting national, state,
  county, and tract-level analysis
\item
  \textbf{Policy relevance}: Direct support for energy assistance
  program evaluation
\end{enumerate}

The package is available from GitHub at
\url{https://github.com/ericscheier/emburden} and is licensed under
AGPL-3+. Documentation, vignettes, and issue tracking are available
through the package website.

\renewcommand\refname{References}
\bibliography{references.bib}



\end{document}
