\documentclass[
]{jss}

%% recommended packages
\usepackage{orcidlink,thumbpdf,lmodern}

\usepackage[utf8]{inputenc}

\author{
Eric Scheier\\Independent Researcher
}
\title{\pkg{emburden}: Temporal Analysis of Household Energy
Burden Using Net Energy Return Metrics}

\Plainauthor{Eric Scheier}
\Plaintitle{emburden: Temporal Analysis of Household Energy
Burden Using Net Energy Return Metrics}
\Shorttitle{\pkg{emburden}: Energy Burden Analysis}


\Abstract{
Energy burden---the proportion of household income spent on energy---is
a critical metric for understanding energy poverty and inequity.
However, traditional energy burden ratios present analytical challenges
including difficulties with aggregation and visualization of extreme
values. The \pkg{emburden} package for \proglang{R} implements
Net Energy Return (Nh) methodology to address these limitations while
enabling temporal analysis of household energy characteristics. This
paper introduces the package's design and demonstrates its application
to comparing Low-Income Energy Affordability Data (LEAD) Tool vintages
from 2018 and 2022 across geographic and demographic dimensions. The
package provides functions for downloading, processing, and analyzing
census tract-level energy burden data for all U.S. states, with
particular attention to proper weighted aggregation and schema
normalization across data vintages. We demonstrate the package's
capabilities through examples ranging from state-level summaries to
fine-grained census tract comparisons, illustrating how policy-relevant
insights can be extracted at multiple scales.
}

\Keywords{energy burden, energy poverty, household energy, net energy
return, temporal analysis, \proglang{R}}
\Plainkeywords{energy burden, energy poverty, household energy, net
energy return, temporal analysis, R}

%% publication information
%% \Volume{50}
%% \Issue{9}
%% \Month{June}
%% \Year{2012}
%% \Submitdate{}
%% \Acceptdate{2012-06-04}

\Address{
    Eric Scheier\\
    Durham, North Carolina\\
  E-mail: \email{eric@scheier.org}\\
  URL: \url{https://github.com/ericscheier}\\~\\
  }


% tightlist command for lists without linebreak
\providecommand{\tightlist}{%
  \setlength{\itemsep}{0pt}\setlength{\parskip}{0pt}}




\usepackage{amsmath}

\begin{document}



\section{Introduction}\label{introduction}

Household energy affordability is a persistent challenge affecting
millions of households in the United States. Low-income households face
disproportionate energy burdens, often spending more than 6\% of their
income on energy costs compared to 2-3\% for higher-income households
\citep{ross2018high, drehobl2016lifting}. Understanding these
disparities and tracking changes over time is essential for designing
effective energy assistance programs and policies.

The traditional energy burden metric---the ratio of energy expenditures
(\(S\)) to gross income (\(G\))---has several analytical limitations. As
a ratio with income in the denominator, energy burden (\(E_b = S/G\))
approaches infinity for households with very low incomes, creating
challenges for aggregation and visualization. Additionally, the metric
requires harmonic mean aggregation rather than arithmetic means, which
is not widely understood or consistently applied
\citep{scheier2022measurement}.

The \pkg{emburden} package for \proglang{R} addresses these
challenges by implementing the Net Energy Return (Nh) transformation:

\[N_h = \frac{G - S}{S}\]

This transformation, inspired by Net Energy Analysis in energy systems
research \citep{hall2011eroi, carbajalesdale2014better}, allows for
proper weighted mean aggregation while preserving the ability to convert
back to energy burden via \(E_b = 1/(N_h + 1)\).

\subsection{The LEAD Tool and temporal
analysis}\label{the-lead-tool-and-temporal-analysis}

The U.S. Department of Energy's Low-Income Energy Affordability Data
(LEAD) Tool \citep{ma2019lowincome} provides census tract-level
estimates of household energy characteristics based on American
Community Survey microdata. The tool uses iterative proportional fitting
to allocate households to census tracts while calibrating to
utility-reported sales and revenues.

Multiple vintages of LEAD Tool data have been released:

\begin{itemize}
\tightlist
\item
  \textbf{2018 Update}: Based on 2018 5-year ACS data, released July
  2020
\item
  \textbf{2022 Update}: Based on 2022 5-year ACS data, released August
  2024
\end{itemize}

These vintages enable temporal analysis of energy burden trends, but
require careful handling of schema differences and income bracket
definitions.

\subsection{Package design philosophy}\label{package-design-philosophy}

The \pkg{emburden} package is designed around several key
principles:

\begin{enumerate}
\def\labelenumi{\arabic{enumi}.}
\tightlist
\item
  \textbf{Proper aggregation}: Implements weighted mean aggregation
  using Net Energy Return, with household counts as weights
\item
  \textbf{Temporal consistency}: Normalizes schema differences between
  LEAD Tool vintages to enable valid comparisons
\item
  \textbf{Flexible workflows}: Supports both database and CSV-based data
  access with automatic fallback
\item
  \textbf{Geographic flexibility}: Enables analysis from national level
  down to individual census tracts
\item
  \textbf{Reproducibility}: All data can be downloaded from public
  sources and processing is fully documented
\end{enumerate}

The remainder of this paper is organized as follows. Section 2 describes
the Net Energy Return methodology and LEAD Tool data structure. Section
3 details the package implementation. Sections 4-6 demonstrate package
capabilities through progressively complex examples. Section 7 discusses
limitations and future extensions.

\section{Methodology}\label{methodology}

\subsection{Net Energy Return
formulas}\label{net-energy-return-formulas}

The Net Energy Return (Nh) of a household with gross income \(G\) and
energy spending \(S\) is defined as:

\[N_h = \frac{G - S}{S} = \frac{G}{S} - 1\]

This can be interpreted as the ratio of net income (after energy costs)
to energy spending. A household with \(N_h = 15.67\) has approximately
\$15.67 of net income per \$1 of energy spending, equivalent to a 6\%
energy burden.

The relationship to energy burden is:

\[E_b = \frac{S}{G} = \frac{1}{N_h + 1}\]

For aggregation across households, the weighted mean Net Energy Return
is:

\[\overline{N_h} = \frac{\sum_{i} w_i N_{h,i}}{\sum_{i} w_i}\]

where \(w_i\) represents household weights (typically household counts
or population). This aggregated value can then be converted to an
aggregate energy burden via \(\overline{E_b} = 1/(\overline{N_h} + 1)\).

\subsection{LEAD Tool data structure}\label{lead-tool-data-structure}

The LEAD Tool provides data at multiple geographic levels (census tract,
county, state) and by multiple income definitions:

\begin{itemize}
\tightlist
\item
  \textbf{AMI}: Area Median Income (income relative to local median)
\item
  \textbf{FPL}: Federal Poverty Line (income relative to poverty
  threshold)
\item
  \textbf{SMI}: State Median Income
\item
  \textbf{LLSI}: Lower Living Standard Income (2022 only)
\end{itemize}

For each household cohort (defined by location, income bracket, housing
tenure, unit type, and other characteristics), the LEAD Tool estimates:

\begin{itemize}
\tightlist
\item
  Number of households
\item
  Mean income
\item
  Mean electricity expenditure
\item
  Mean gas expenditure
\item
  Mean other fuel expenditure
\end{itemize}

These estimates are based on ACS microdata calibrated to
utility-reported totals.

\subsection{Schema differences between
vintages}\label{schema-differences-between-vintages}

The 2018 and 2022 LEAD Tool releases have significant schema
differences:

\textbf{Income brackets}: - 2018 AMI: 5 brackets (0-30\%, 30-50\%,
50-80\%, 80-100\%, 100\%+) - 2022 AMI: 6 brackets (0-30\%, 30-60\%,
60-80\%, 80-100\%, 100-150\%, 150\%+)

\textbf{Column structure}: - 2018: Separate columns for each attribute
(TEN, YBL6, BLD, HFL) - 2022: Combined columns (TEN-YBL6, TEN-BLD,
TEN-HFL)

\textbf{New features in 2022}: - 12 demographic columns - LLSI income
metric - Tribal area geographies - Frequency weights

The \pkg{emburden} package handles these differences through
schema normalization, mapping income brackets to common categories and
parsing combined columns.

\section{Package implementation}\label{package-implementation}

\subsection{Package structure}\label{package-structure}

The \pkg{emburden} package is organized into several functional
modules:

\begin{itemize}
\tightlist
\item
  \textbf{Energy metrics} (\texttt{energy\_ratios.R}): Core calculations
  for Nh, EROI, DEAR
\item
  \textbf{Data loading} (\texttt{lead\_data\_loaders.R},
  \texttt{csv\_fallback.R}): Download and import LEAD data
\item
  \textbf{Database integration} (\texttt{emrgi\_data\_loaders.R}): Query
  SQLite database
\item
  \textbf{Temporal comparison} (\texttt{compare\_burden.R}): Compare
  vintages with normalization using \texttt{compare\_energy\_burden()}
\item
  \textbf{Statistical analysis} (\texttt{metrics.R}): Weighted
  aggregation functions
\item
  \textbf{Formatting} (\texttt{formatting.R}): Output formatting for
  tables and reports
\end{itemize}

\subsection{Core functions}\label{core-functions}

\subsubsection{Data acquisition}\label{data-acquisition}

The package provides functions to download LEAD Tool data directly from
OpenEI:

\begin{CodeChunk}
\begin{CodeInput}
R> library("emburden")
R> 
R> # Download 2022 data for North Carolina
R> files_2022 <- download_lead_data_from_openei(
+   vintage = "2022",
+   states = "NC"
+ )
R> 
R> # Process AMI census tract data
R> nc_ami <- process_lead_cohort_data(
+   file_path = files_2022$NC["ami_tract"],
+   vintage = "2022",
+   income_metric = "ami"
+ )
\end{CodeInput}
\end{CodeChunk}

\subsubsection{Data loading with
fallback}\label{data-loading-with-fallback}

For routine analysis, higher-level functions provide automatic
database/CSV fallback:

\begin{CodeChunk}
\begin{CodeInput}
R> # Load latest data (tries database, falls back to CSV)
R> nc_data <- load_cohort_data(
+   dataset = "ami",
+   states = "NC"
+ )
R> 
R> # Load specific vintage
R> nc_2018 <- load_cohort_data(
+   dataset = "ami",
+   states = "NC",
+   vintage = "2018"
+ )
\end{CodeInput}
\end{CodeChunk}

\subsubsection{Temporal comparison}\label{temporal-comparison}

The core comparison function handles schema normalization and
aggregation. The \texttt{compare\_energy\_burden()} function compares
energy burden across vintages using proper aggregation methodology.
For cohort data (pre-aggregated households), the function sums totals
first, then calculates ratios: \(NEB = \sum S_i / \sum G_i\). This
avoids division-by-zero issues with row-by-row calculations.

\begin{CodeChunk}
\begin{CodeInput}
R> # Compare by income bracket (2018 vs 2022)
R> comparison <- compare_energy_burden(
+   dataset = "ami",
+   states = "NC",
+   group_by = "income_bracket",
+   vintage_1 = "2018",
+   vintage_2 = "2022",
+   format = TRUE
+ )
R>
R> # View formatted results
R> print(comparison)
\end{CodeInput}
\end{CodeChunk}

\subsection{Design decisions}\label{design-decisions}

Several key design decisions shape the package architecture:

\begin{enumerate}
\def\labelenumi{\arabic{enumi}.}
\tightlist
\item
  \textbf{Lazy evaluation}: Data is not downloaded/loaded until
  explicitly requested
\item
  \textbf{Graceful degradation}: Database unavailability falls back to
  CSV
\item
  \textbf{Explicit vintage specification}: Prevents accidental mixing of
  vintages
\item
  \textbf{Comprehensive metadata}: All data includes vintage and source
  information
\item
  \textbf{Memory efficiency}: State-by-state processing for large
  analyses
\end{enumerate}

\section{Basic state-level
comparison}\label{basic-state-level-comparison}

We begin with a simple state-level comparison to illustrate basic
package usage.

\begin{CodeChunk}
\begin{CodeInput}
R> library("emburden")
R> library("dplyr")
R> 
R> # Compare North Carolina: 2018 vs 2022
R> nc_state <- compare_vintages(
+   dataset = "ami",
+   states = "NC",
+   aggregate_by = "state"
+ )
R> 
R> # Calculate energy burdens
R> nc_state <- nc_state %>%
+   mutate(
+     burden_2018 = (total_electricity_spend_2018 +
+                    total_gas_spend_2018 +
+                    total_other_spend_2018) / total_income_2018,
+     burden_2022 = (total_electricity_spend_2022 +
+                    total_gas_spend_2022 +
+                    total_other_spend_2022) / total_income_2022,
+     burden_change_pp = (burden_2022 - burden_2018) * 100,
+     burden_change_pct = (burden_change_pp / (burden_2018 * 100)) * 100
+   )
R> 
R> # Display results
R> print(nc_state[, c("state", "burden_2018", "burden_2022",
+                    "burden_change_pp", "burden_change_pct")])
\end{CodeInput}
\end{CodeChunk}

The output shows the aggregate energy burden for North Carolina
decreased from X\% in 2018 to Y\% in 2022, representing a relative
change of Z\%.

\section{Income bracket analysis}\label{income-bracket-analysis}

Energy burden varies dramatically by income level. We demonstrate income
bracket comparison:

\begin{CodeChunk}
\begin{CodeInput}
R> # Compare by income bracket
R> nc_income <- compare_vintages(
+   dataset = "ami",
+   states = "NC",
+   aggregate_by = "income_bracket"
+ )
R> 
R> # Calculate burdens by bracket
R> nc_income <- nc_income %>%
+   mutate(
+     burden_2018 = (total_electricity_spend_2018 +
+                    total_gas_spend_2018 +
+                    total_other_spend_2018) / total_income_2018,
+     burden_2022 = (total_electricity_spend_2022 +
+                    total_gas_spend_2022 +
+                    total_other_spend_2022) / total_income_2022
+   ) %>%
+   arrange(income_bracket)
R> 
R> # Visualize
R> library("ggplot2")
R> 
R> ggplot(nc_income, aes(x = income_bracket)) +
+   geom_col(aes(y = burden_2018 * 100, fill = "2018"),
+            position = "dodge", alpha = 0.7) +
+   geom_col(aes(y = burden_2022 * 100, fill = "2022"),
+            position = "dodge", alpha = 0.7) +
+   labs(
+     title = "Energy Burden by Income Bracket: 2018 vs 2022",
+     subtitle = "North Carolina",
+     x = "Income Bracket (% of Area Median Income)",
+     y = "Energy Burden (%)",
+     fill = "Year"
+   ) +
+   theme_minimal() +
+   theme(axis.text.x = element_text(angle = 45, hjust = 1))
\end{CodeInput}
\end{CodeChunk}

This analysis reveals which income groups experienced the largest
changes in energy burden, informing targeted policy interventions.

\section{Census tract-level analysis}\label{census-tract-level-analysis}

For local policy and program design, census tract-level analysis
identifies specific communities with high burdens or large changes:

\begin{CodeChunk}
\begin{CodeInput}
R> # Get tract-level comparison
R> nc_tracts <- compare_vintages(
+   dataset = "ami",
+   states = "NC",
+   aggregate_by = "tract"
+ )
R> 
R> # Calculate burden changes
R> nc_tracts <- nc_tracts %>%
+   mutate(
+     burden_2018 = (total_electricity_spend_2018 +
+                    total_gas_spend_2018 +
+                    total_other_spend_2018) / total_income_2018,
+     burden_2022 = (total_electricity_spend_2022 +
+                    total_gas_spend_2022 +
+                    total_other_spend_2022) / total_income_2022,
+     burden_change = burden_2022 - burden_2018
+   )
R> 
R> # Identify tracts with largest increases
R> worst_changes <- nc_tracts %>%
+   filter(burden_change > 0) %>%
+   arrange(desc(burden_change)) %>%
+   head(10)
R> 
R> print(worst_changes[, c("geoid", "burden_2018", "burden_2022",
+                          "burden_change", "households_2022")])
\end{CodeInput}
\end{CodeChunk}

These tract-level results can be joined with census geography for
mapping or with demographic data for further analysis.

\section{Multi-state regional
comparison}\label{multi-state-regional-comparison}

The package efficiently handles multi-state analyses for regional
comparisons:

\begin{CodeChunk}
\begin{CodeInput}
R> # Compare Southern states
R> southern <- c("NC", "SC", "GA", "VA", "TN", "FL", "AL", "MS", "LA", "AR")
R> 
R> regional <- compare_vintages(
+   dataset = "ami",
+   states = southern,
+   aggregate_by = "state"
+ ) %>%
+   mutate(
+     burden_2018 = (total_electricity_spend_2018 +
+                    total_gas_spend_2018 +
+                    total_other_spend_2018) / total_income_2018,
+     burden_2022 = (total_electricity_spend_2022 +
+                    total_gas_spend_2022 +
+                    total_other_spend_2022) / total_income_2022,
+     burden_change = burden_2022 - burden_2018
+   ) %>%
+   arrange(desc(burden_change))
R> 
R> # Visualize regional patterns
R> ggplot(regional, aes(x = reorder(state, burden_change))) +
+   geom_col(aes(y = burden_change * 100), fill = "steelblue") +
+   geom_hline(yintercept = 0, linetype = "dashed") +
+   coord_flip() +
+   labs(
+     title = "Change in Energy Burden: 2018 to 2022",
+     subtitle = "Southern States",
+     x = "State",
+     y = "Change (percentage points)"
+   ) +
+   theme_minimal()
\end{CodeInput}
\end{CodeChunk}

\section{Discussion and limitations}\label{discussion-and-limitations}

\subsection{Limitations}\label{limitations}

Several limitations should be considered when using this package:

\begin{enumerate}
\def\labelenumi{\arabic{enumi}.}
\tightlist
\item
  \textbf{Data quality}: LEAD Tool estimates are based on statistical
  modeling and inherit uncertainties from ACS microdata and utility
  reporting
\item
  \textbf{Income bracket changes}: The different bracket definitions
  between 2018 and 2022 require aggregation for exact comparison
\item
  \textbf{Missing tracts}: Some census tracts present in one vintage may
  be absent in another due to boundary changes
\item
  \textbf{Temporal scope}: Only two comparison points (2018, 2022) are
  currently available
\end{enumerate}

\subsection{Future extensions}\label{future-extensions}

Planned enhancements include:

\begin{itemize}
\tightlist
\item
  Unit tests for all core functions
\item
  Integration of additional data sources (utility rates, emissions,
  weather)
\item
  Spatial analysis functions using \pkg{sf}
\item
  Time series visualization when more vintages become available
\item
  Statistical significance testing for changes
\end{itemize}

\section{Summary}\label{summary}

The \pkg{emburden} package provides comprehensive tools for
analyzing household energy burden using Net Energy Return methodology.
The package handles the complexity of temporal comparisons across LEAD
Tool vintages while enabling analysis at multiple geographic and
demographic scales. By implementing proper aggregation methods and
schema normalization, the package facilitates policy-relevant research
on energy poverty and inequity.

\section{Acknowledgments}\label{acknowledgments}

This work builds on the foundational LEAD Tool developed by the U.S.
Department of Energy and the Nature Communications paper co-authored
with Noah Kittner. The author thanks the reviewers for helpful comments.

\renewcommand\refname{References}
\bibliography{jss-netenergyequity.bib}



\end{document}
